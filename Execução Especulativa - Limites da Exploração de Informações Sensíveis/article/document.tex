\documentclass[conference]{IEEEtran}
\IEEEoverridecommandlockouts
% The preceding line is only needed to identify funding in the first footnote. If that is unneeded, please comment it out.
\usepackage{cite}
\usepackage{amsmath,amssymb,amsfonts}
\usepackage{algorithmic}
\usepackage{graphicx}
\usepackage{textcomp}
\usepackage{xcolor}

\usepackage[brazil,english]{babel}
\usepackage[utf8]{inputenc}
\usepackage{hyperref}

\def\BibTeX{{\rm B\kern-.05em{\sc i\kern-.025em b}\kern-.08em
    T\kern-.1667em\lower.7ex\hbox{E}\kern-.125emX}}

%\renewcommand\IEEEkeywordsname{Palavras-chave}

\begin{document}
\selectlanguage{brazil}

\title{Execução Especulativa:\\
Limites da Exploração de Informações Sensíveis}

\author{
    \IEEEauthorblockN{Felipe R. da S. Cordeiro, Éldman de O. Nunes}
    \IEEEauthorblockA{\textit{Escola de Arquitetura, Engenharia e Tecnologia da Informação (EAETI)}\\
    \textit{Universidade Salvador (UNIFACS)}\\
    Salvador, BA, Brasil\\
    \href{mailto:felipe.rios@outlook.com.br}{felipe.rios@outlook.com.br}, \href{mailto:eldman.nunes@unifacs.br}{eldman.nunes@unifacs.br}}
}

\maketitle

\begin{abstract}
As falhas de segurança se tornam mais preocupantes quando a segurança de informações pessoais correm risco. As descobertas de Jann Horn e de outros grupos de pesquisadores colocaram a prova a segurança de vários processadores modernos. Batizada de \emph{Spectre}, a vulnerabilidade que se aproveita da técnica de execução especulativa, está presente em quase todos os dispositivos computacionais que executam funções ao mesmo tempo. Mas, até que ponto as informações pessoais dos usuários estão expostas? Quais são os limites da exploração desta vulnerabilidade? Utilizando procedimentos técnicos e testes fundamentados, esta pesquisa esclarece os limites fundamentais que precisam ser atendidos para um vazamento ocorrer através da exploração desta vulnerabilidade. Com a documentação e o código criado pelos pesquisadores supracitados, possíveis limites de exploração da vulnerabilidade foram elicitados e provados. Foi possível delimitar as circunstâncias necessárias para a exploração da primeira variação da \emph{Spectre} em código nativo.
\end{abstract}

\begin{IEEEkeywords}
execução, especulativa, \emph{Spectre}, limites, exploração, informações
\end{IEEEkeywords}

% ------------------------------------------------------------------------
% INTRODUÇÃO
% ------------------------------------------------------------------------
\section{Introdução}
A execução especulativa é uma técnica de projeto de microarquitetura, que proporciona o aprimoramento da velocidade de processamento nas CPU's modernas. Esta técnica consiste na estimativa e execução de instruções com valores ainda não conhecidos pela CPU, durante o curto período de carregamento dos valores reais. Do ponto de vista funcional, esta especulação traria problemas se os resultados de especulações incorretas fossem efetivados. Porém, quando a verdadeira informação é recuperada, a CPU verifica a exatidão da suposição e descarta o ``caminho'' (fluxo de execução) que foi executado incorretamente, eliminando os valores nos registradores ou alterações em variáveis, por exemplo \cite{Intel2016Architectures}.

Utilizando ataques que combinam a indução da execução especulativa e canais laterais, Jann Horn (\citeyear{Jann2018Reading}) e outros pesquisadores provaram que é possível recuperar informações privilegiadas (que foram especuladas), através da memória cache (um canal lateral), que não é revertida quando uma especulação errônea acontece \cite{Kocher2018Spectre}. A partir disto, com o conhecimento microarquitetural necessário, é possível um atacante induzir a execução especulativa errônea, por ``viciar'' o processador em uma cadeia de especulações e transferir as informações especuladas para um canal alternativo que pode ser lido posteriormente.

A abrangência de processadores atingidos, torna esta vulnerabilidade um assunto relevante devido a probabilidade de que muitas máquinas, aparelhos móveis, \emph{datacenters} e consequentemente informações sigilosas estejam expostas. Isto acontece porque, tal categoria de vulnerabilidade se encontra em nível de \emph{hardware} e correções via \emph{software} são paliativos que custam questões de desempenho e estabilidade.

Considerando que os pesquisadores citados utilizaram \emph{softwares} com estruturas conhecidas (endereços de memória, tamanho e conteúdo das variáveis por exemplo) em seus experimentos, algumas questões ainda permanecem em aberto. Diante disto, é possível explorar informações sensíveis de \emph{softwares} cuja estrutura de memória seja desconhecida, por meio da exploração da execução especulativa induzida, utilizando canais laterais?

Através deste questionamento, esta pesquisa visa esclarecer, explicar e aplicar técnicas desenvolvidas pelos pesquisadores supramencionados. Visando elencar e comprovar as circunstâncias envolvidas e que realmente são necessárias para se executar vazamentos de informações sensíveis. Tendo como foco, a exploração da execução especulativa em \emph{softwares} cujas estruturas se desconhece. De forma a solidificar limitações desta técnica de exploração em particular.

% ------------------------------------------------------------------------
% FUNDAMENTAÇÃO TEÓRICA
% ------------------------------------------------------------------------
\section{Fundamentação Teórica}
% ---------------------------------
% EXECUÇÃO FORA DE ORDEM
% ---------------------------------
\subsection{Execução Fora de Ordem}
As funções que se encontram ordenadas em um fluxo para execução de um \emph{software} qualquer, de forma geral, são convertidas pelos processadores para instruções. E, dentro do núcleo do processador, estas instruções por sua vez, são convertidas para micro-operações \cite{Alisson2017Introducao}. A execução destas instruções pelo processador, acontece fora de ordem em grande parte dos momentos. Este paradigma é chamado de \textbf{execução fora de ordem} (\emph{out-of-order execution}) \cite{Fog2017Microarchitecture}.

É um paradigma que aumenta a utilização dos componentes do processador. Permite que instruções que sucedem o fluxo atual de execução, sejam executadas em paralelo, ou até mesmo antes do fluxo de execução atual. E, assim que as micro-operações de uma instrução são finalizadas, bem como as instruções que a antecedem, estas instruções são retiradas, obedecendo a ordem de execução predefinida. Limpando assim o \emph{buffer} de reordenação e efetivando as alterações nos registradores (variáveis do processador) \cite{Kocher2018Spectre}.

% ---------------------------------
% DESVIO CONDICIONAL
% ---------------------------------
\subsection{Desvio Condicional}
Durante o fluxo de execução deste \emph{software} qualquer, utilizado como exemplo, podem existir instruções de desvio: condicionais ou incondicionais. Quando o processador se depara com uma instrução de desvio condicional (chamada de \textbf{\emph{branch condition}}), muitas vezes ele não conhece quais são as instruções futuras. Pois, esta instrução de ramificação tem uma direção que depende das instruções precedentes, cuja execução ainda não está concluída. 

Esta situação é claramente vista quando observa-se o endereço de memória que aponta para a próxima instrução (que continuaria o fluxo). Este endereço pode ser especificado de forma: imediata, endereço de memória fixo; direta, variável que contém o endereço de memória; ou indireta, endereço de memória de uma variável que contém o próximo salto \cite{Debarshi2018Addressing}. Quando o endereço de memória é especificado através de uma variável (de forma direta) e o conteúdo desta variável se trata do resultado de uma operação, tem-se uma condição propícia para uma ``especulação''.

% ---------------------------------
% EXECUÇÃO ESPECULATIVA
% ---------------------------------
\subsection{Execução Especulativa}
Nestas situações, o processador pode preservar o estado dos registradores e variáveis, fazer uma previsão do caminho que o programa seguirá e executar instruções especulativamente neste caminho. Se a previsão estiver correta, os resultados desta execução serão confirmados (isto é, salvos nos registradores e variáveis), gerando uma vantagem de desempenho em relação ao tempo ocioso durante a espera. Caso contrário, o processador abandona o trabalho executado, revertendo o estado dos registradores e variáveis, continuando ao longo do caminho correto. Esta técnica é chamada de \textbf{execução especulativa} (\emph{speculative execution}) \cite{Kocher2018Spectre}.

% ---------------------------------
% PREDIÇÃO E INSTRUÇÕES TRANSIENTES
% ---------------------------------
\subsection{Predição e Instruções Transientes}
Durante a execução especulativa, o processador faz suposições sobre o resultado provável das instruções de desvio. Previsões melhores consequentemente melhoram o desempenho, aumentando o número de operações executadas especulativamente que podem ser confirmadas com sucesso \cite{Kocher2018Spectre}. Esta técnica de previsão de resultados, aplicada a uma \emph{branch condition} é chamada de \textbf{\emph{branch prediction}}.

\citeonline{Kocher2018Spectre} chamou as instruções que são realizadas erroneamente em uma tentativa de execução especulativa (como resultado de uma previsão incorreta) de ``instruções transitórias'' (\textbf{\emph{transient instructions}}). Admitindo que tais instruções, embora não alterem o fluxo de execução de um \emph{software}, deixam rastros em componentes micro arquiteturais. Quando um invasor passa a coletar esses rastros através destes componentes, pode-se dizer que estes dados estão vazando através de um \textbf{canal}.

% ---------------------------------
% ATAQUES DE CANAIS LATERAIS
% ---------------------------------
\subsection{Ataques de Canais Laterais}
O comportamento de prever uma ação futura, baseado em ações passadas, assumindo que a ação futura é semelhante a anterior, permite que alterações no estado de micro arquitetura causadas pelo comportamento de um \emph{software} possam afetar outros \emph{softwares}. E, ao longo do tempo, a exploração destas pequenas alterações foram comprovadas de diferentes formas, elicitadas por \citeonline{Kocher2018Spectre}. As explorações de componentes micro arquiteturais (canais), são chamadas de \textbf{ataques de canais laterais} (\emph{side-channels attacks}). E, um destes canais que é conhecido e costumeiramente explorado por ataques de canais laterais, é a memória cache.

% ------------------------------------------------------------------------
% TRABALHOS RELACIONADOS
% ------------------------------------------------------------------------
\section{Trabalhos Relacionados}
As descobertas supramencionadas abriram precedentes para outros questionamentos em relação a segurança dos processadores atuais. Questionamentos sobre quantas vulnerabilidades críticas como estas, podem estar ``adormecidas'' por anos, preocupam os pesquisadores \cite{Andy2018Triple}. E motivados por estas preocupações, outras vulnerabilidades foram descobertas, partindo do mesmo princípio de ataques que se aproveitam da execução especulativa. Em ordem cronológica, foram documentadas: \emph{Foreshadow} em \citeonline{VanBulck2018Foreshadow} e \emph{Foreshadow-NG} em \citeonline{Weisse2018ForeshadowNG}. \emph{PortSmash} em \citeonline{Alejandro2018Port}. \emph{Spoiler}, em \citeonline{Islam2019Spoiler}. \emph{ZombieLoad} em \citeonline{Schwarz2019ZombieLoad}. \emph{RIDL} em \citeonline{VanSchaik2019RIDL} e \emph{Fallout} em \citeonline{Minkin2019Fallout}.

A \emph{Spectre}, que dada a cronologia de descoberta pode ser chamada de ``pioneira'' da execução especulativa, é complexa de ser explorada e possui circunstâncias (ou condições) específicas para exploração. Por isso, pesquisadores continuam a desenvolver análises e outras variações de ataques utilizando a \emph{Spectre} como base. Exemplo recente disto é: a análise publicada por \citeonline{McIlroy2019SpectreIH}; e a variação de ataque utilizando o RSB (\emph{Return Stack Buffer}) como canal micro arquitetural, que foi documentada por \citeonline{Esmaeil2018Spectre}.

Até o momento de finalização desta pesquisa (junho de 2019), não havia sido descoberta uma solução totalmente eficaz contra a vulnerabilidade. Um grande esforço estava sendo feito para o desenvolvimento de medidas de prevenção via \emph{software}, para dificultar a exploração e aumentar a robustez das aplicações \cite{Graz2018Meltdown}.

% ------------------------------------------------------------------------
% METODOLOGIA
% ------------------------------------------------------------------------
\section{Metodologia}
Para a realização dos experimentos, utilizou-se a prova de conceito (\emph{Proof of Concept}, ou \emph{PoC}) documentada por \citeonline{Kocher2018Spectre}. O código exemplo de exploração (ou \emph{exploit}\footnote{É um conjunto de instruções \cite{Cruz2016Estudo} capaz de tirar proveito de uma vulnerabilidade \cite{Cambridge2019Exploit}.}) desta \emph{PoC} foi escrito em linguagem C e simula o vazamento de uma informação secreta escrita diretamente no código. O intuito é explorar a execução especulativa induzida, utilizando a memória cache como canal lateral em uma estrutura (endereços de memória e variáveis) definida pelo próprio criador do \emph{exploit}. 

De posse deste \emph{exploit} e utilizando uma máquina convencional que contenha um compilador de C (\textbf{Visual C++} para \emph{Windows}, ou \textbf{GNU Compiler Collection} para \emph{Linux}), é possível verificar os passos que desencadeiam o vazamento. No caso desta pesquisa, o ambiente utilizado foi um Linux GNU Ubuntu 16.04.05 LTS 64-bit, Intel\textsuperscript{\tiny\textregistered} Core\textsuperscript{\tiny\texttrademark} i3 5005U @ 2.0 Ghz x 4. Diante disto, os experimentos para a realização desta pesquisa foram divididos em quatro testes (T01 à T04), que apresentam-se como possíveis circunstâncias limitadoras para a exploração da \emph{Spectre}. Tendo como foco, a utilização do \emph{exploit} em uma estrutura contrária ao documentado por \citeonline{Kocher2018Spectre}: endereços de memória desconhecidos, tamanho e conteúdo das variáveis escolhidos de forma adversa.

Dois métodos foram utilizados para elicitar estes \textbf{possíveis limites} (ou circunstâncias limitadoras): exploração da literatura, em busca de circunstancias que sejam necessárias para a execução da \emph{Spectre}; e, exploração do \emph{exploit}, em busca de limitações intuitivas que determinam o funcionamento da \emph{PoC}. Posteriormente estes foram testados, afim de identificar com as suas possíveis falhas, em quais circunstâncias gerais e de comum aceitação a \emph{Spectre} funciona ou limita-se: quanto a endereços de memória (físicos ou virtuais) que abrangem o vazamento, quanto a quantidade de dados que pode ser vazada e quanto a aplicação genérica da vulnerabilidade para leitura arbitrária de memória, sem autorização.

% ------------------------------------------------------------------------
% EXPERIMENTOS
% ------------------------------------------------------------------------
\section{Experimentos}
Esta seção tem por objetivo expor os testes realizados de maneira sistemática, com seus resultados individuais, abrindo questionamentos a respeito do conhecimento estudado. Ao final desta seção, uma discussão apresenta uma síntese do produto gerado pelos experimentos.
% ---------------------------------
% T01
% ---------------------------------
\subsection{T01 - Fora de contexto com endereço físico}
Conforme \citeonline{Kocher2018Spectre}, os ataques utilizando \emph{Spectre} são efetivos para as informações que a vítima tem acesso. Este tipo de ataque não escala privilégios e acessa informações privilegiadas (dependendo das informações que a vítima tem acesso). Mesmo assim, a literatura referencial afirma que, limitando-se apenas aos dados que o contexto da vítima tem acesso, é possível realizar vazamentos de dados que estão fora do contexto do atacante. Assim, o objetivo deste experimento é comprovar que o \emph{exploit} da \emph{Spectre} pode realizar estes vazamentos.

O apêndice \ref{appendix:a} apresenta o código, adaptado pelos autores, empregado para testar o vazamento de dados fora do contexto do atacante. Expondo de forma voluntária o seu próprio PID (\emph{Process ID}, identificador único do processo para o Sistema Operacional) e o endereço virtual da variável que contém as informações secretas. Esta exposição voluntária foi necessária para facilitar a busca das variáveis entre os diferentes contextos. Com o intuito de não haver maiores empecilhos na recuperação dos dados.

No apêndice \ref{appendix:a} também encontra-se uma adaptação do \emph{exploit} original\footnote{O \emph{exploit} original documentado por \citeonline{Kocher2018Spectre} encontra-se em anexo ao referido documento.} voltado para o ataque. A adaptação feita utilizou funções extras para converter endereços virtuais de memória para físicos\footnote{O código original desta função pode ser encontrado em: \url{https://github.com/cirosantilli/linux-kernel-module-cheat/blob/873737bd1fc6e5ee0378f21e9df1c52e3f61e3fb/userland/common.h}} e limpar a memória cache com determinado tamanho (técnica abordado no teste 02). A partir disto, a aplicação vítima entrega o seu próprio \emph{PID} e o endereço virtual da informação secreta. Esses dados sensíveis foram inseridos na aplicação atacante, com o intuito de substituir o cálculo de endereçamento inicial por um endereço físico baseado nos mapeamentos de memória do Linux \cite{EQWARE2009Capturing}. A figura \ref{lst:01} mostra o código que precisa ser inserido no lugar das linhas 175 a 188 na aplicação atacante do apêndice \ref{appendix:a}. Em que as duas sequências de ``000'' são \emph{PID} e endereço virtual da variável que contém a informação secreta, respectivamente. Com estas alterações feitas, as inferências foram realizadas com até 1 MB de diferença a frente deste endereço físico (margem de erro considerável, para localizar o conteúdo da variável).

Com o objetivo de não haver divergências nos resultados entre testes, as referências a função \lstinline[language=C, style=c]{_mm_clflush} foram mantidas as mesmas conforme o \emph{exploit} original. Diferentemente do segundo teste desta pesquisa, que visa testar outro aspecto do ataque. Da mesma forma, o código vulnerável do \emph{exploit} original também foi mantido nesta aplicação atacante para este teste em específico. Não houve chamadas diretas a uma aplicação externa, visto que se quis testar a capacidade do código vulnerável explorar um contexto diferente do originário, sendo os endereços físicos da memória os alvos dos ``ataques'' à vítima.

\lstinputlisting[language=C, style=c, caption={Linhas modificadas na aplicação atacante.}, label={lst:01}, firstnumber=175]{listings/list01.c}

Após o termino das inferências (5 minutos), foi possível perceber que não é possível inferir dados de outro contexto, utilizando o endereço físico como usuário normal. Porque primeiro, o sistema operacional limita o acesso mesmo que seja de somente leitura a outras variáveis de outros contextos (o isolamento de memória padrão do sistema pode ser a possível causa). E segundo, é preciso utilizar privilégios superiores aos de usuário normal para adquirir o endereço físico de uma variável qualquer de outro processo a partir dos mapeamentos de memória do Linux \cite{EQWARE2009Capturing}.

% ---------------------------------
% T02
% ---------------------------------
\subsection{T02 - Fora de contexto com \emph{flush} da cache}
Usando ainda a afirmação de \citeonline{Kocher2018Spectre}, a respeito da execução da \emph{Spectre} em contextos distintos, foi possível elencar outra estratégia de execução do \emph{exploit} em outros contextos. O objetivo deste experimento é testar outra possibilidade de vazamento utilizando contextos diferentes, porém, aplicando a técnica de Carga+Descarga (\emph{Flush+Reload}, conforme o artigo original) do cache.

O código atacante do apêndice \ref{appendix:a}, foi utilizado na íntegra (a não ser a modificação do \emph{PID} e do endereço virtual corretos). Porém, a figura \ref{lst:04} do apêndice \ref{appendix:a} mostra um código de vítima modificado para receber por argumento de inicialização variáveis que podem estourar sua estrutura condicional vulnerável. E esta modificação caracteriza as escritas na memória cache. Por sua vez o atacante mapeou os endereços físicos de duas variáveis do espaço da vítima, afim de capturar a diferença entre o vetor de exploração e o vetor secreto de caracteres. Com esta diferença em mãos, era esperado estourar o vetor de exploração e atingir caracteres secretos sem provocar sinais de \emph{SIGSEGV} na execução.

No locais onde a instrução \lstinline[language=C, style=c]{_mm_clflush} era utilizada, foi inserida uma função para realizar \emph{Flush+Reload} da memória cache, com instruções em \emph{Assembly}\footnote{O código original desta função pode ser encontrado em: \url{https://github.com/lemire/Code-used-on-Daniel-Lemire-s-blog/blob/master/2018/01/04/align.c}}. Espera-se efetuar o \emph{flush} de endereços físicos do contexto da vítima, para a concretização da técnica, pois as chamadas exaustivas que provocaram a recarga das informações estão funcionando teoricamente.

Após os testes, a aplicação atacante parou sua execução com um sinal de \emph{SIGSERV (Segmentation fault}, ou falha de segmentação). Acusando a não permissão de realização do \emph{flush} de um endereço físico, fora do próprio contexto. Isto acontece pois, segundo \emph{Intrinsics Guide} da \emph{Intel}, instruções de \emph{flush} são utilizadas para limpar uma informação de mesmo contexto, em todos os níveis da hierarquia da cache \cite{Intel2018Intrinsics}. E, ao especular os dados o \emph{exploit} precisa fazer o ciclo de tentativa de acesso e limpeza da cache sucessivas vezes, inferindo o menor tempo de acesso das variáveis. Quando isto não ocorre no mesmo contexto, o \emph{exploit} não é capaz de manipular o comportamento da memória cache e portanto, não consegue realizar as inferências.

\begin{comment}
explicar mais sobre o que é "contexto" nesse sentindo
\end{comment}

% ---------------------------------
% T03
% ---------------------------------
\subsection{T03 - Limite de tamanho da cache}
As especificações da \emph{Intel} (\citeyear{Intel2019Corei3}) e \emph{CPU Wolrd} (\citeyear{CPU2016Corei3}), documentam o tamanho da hierarquia da memória cache, utilizada neste experimento. É possível confirmar estes valores utilizando o comando\footnote{Documentado nas páginas do manual do \emph{Ubuntu} (\citeyear{Ubuntu2019Lscpu}).}: \lstinline[language=C, style=c]{lscpu | grep "cache"}. Que tem como resultado os valores de cada unidade na hierarquia: \textbf{L1d: 32K} (dados), \textbf{L1i: 32K} (instruções), \textbf{L2: 256K}, \textbf{L3: 3072K}. Objetiva-se comprovar que é possível utilizar o \emph{exploit} da \emph{Spectre}, fora dos limites que a memória cache do processador em questão é capaz de suportar. Aplicando tamanhos e conteúdos dinâmicos nas variáveis. Diferente do modelo original no código do \emph{exploit}, que contém variáveis com o conteúdo estático e tamanho fixo. Será levado em conta o tamanho da cache \emph{LLC}, pois o \emph{branch predictor} pode fazer um cache \emph{hit} em qualquer nível de cache.

A figura \ref{lst:05} mostra o código de adaptação que foi utilizado para calcular o tempo de execução do \emph{exploit} a partir do inicio da fase de treinamento. Após isto, a primeira modificação pontual feita para este experimento, foi o aumento do tamanho da informação secreta no código do \emph{exploit}. Aumentado de 40 bytes, para um valor randômico entre 3072 e 4072 Kbytes (conforme o código na figura \ref{lst:05}). Depois, este vetor de caracteres foi preenchido randomicamente e submetido ao código do \emph{exploit} para inferência das variáveis. Era esperado que a memória cache removesse os valores iniciais e somente os valores finais pudessem ser inferidos.

\lstinputlisting[language=C, style=c, caption={Contagem do tempo de execução e geração do vetor randômico.}, label={lst:05}]{listings/list05.c}

Após a finalização deste experimento, concluiu-se que, ao saber o ponteiro de inicio da variável que contém a informação secreta e de posse do tamanho desta informação, é possível inferir todos os dados em uma velocidade que está num intervalo de \textbf{4 KB/s} a \textbf{7 KB/s} (para o ambiente em questão). Mesmo que a informação inferida seja maior que a memória cache. Isto porque, a transferência dos dados acontece com um endereço por vez. Cada endereço inferido passa por 3 fases: treinamento (\emph{loop} que acontece 30 vezes), leitura (\emph{loop} de 256 vezes) e inferência da melhor leitura (\emph{loop} de 256 vezes). Sendo que, cada etapa ocorre 999 vezes, garantindo uma \textbf{acurácia de 0,01\%} nos resultados, conforme \citeonline{Kocher2018Spectre}.

\begin{comment}
Adicionar uma imagem com as saídas.
\end{comment}

% ---------------------------------
% T04
% ---------------------------------
\subsection{T04 - Utilização de \emph{bit twiddling}}
A figura \ref{lst:06} contém um trecho do \emph{exploit} original chamado de \emph{bit twiddling} pelo seus autores. Este trecho de código está presente no \emph{exploit}, para unicamente simular o funcionamento de uma estrutura condicional. Segundo \citeonline{Kocher2018Spectre}, as estruturas condicionais ativam o \emph{branch predictor} e, durante a fase de treinamento falso (\emph{mistraining}, ou treinamento errado) da memória cache, o \emph{predictor} não deveria ser chamado, pois adicionaria mais linhas na cache e, de certa forma, agregaria imprecisão a inferência do \emph{exploit}.

\lstinputlisting[language=C, style=c, caption={\emph{Bit twiddling} no código do \emph{exploit}.}, label={lst:06}, firstnumber=58]{listings/list06.c}

Trocando esta estrutura de \emph{bit twiddling}, para uma estrutura condicional convencional (teste se o resto da divisão de ``j'' por 6 for 0, o endereço maliciosamente escolhido é aplicado), espera-se provar que o grau de imprecisão agregado ao \emph{exploit} não é relevante para a checagem das informações. Sendo esta troca bem sucedida, tem-se uma variação do \emph{exploit} mais legível e fácil de se interpretar.

Concluiu-se que, após a troca das estruturas condicionais, ao ativar o \emph{branch predictor} na fase de treinamento, o \emph{buffer} que ordena os endereços nas linhas de cache reordenou as referências, o que afetou a inferência de dados do \emph{exploit}. Também foi percebido que isso só acontece no contexto atual. Pois quando o \emph{exploit} foi testado, outras atividades estavam ocorrendo na máquina alvo e isto não afetou as inferências de informações.

% ------------------------------------------------------------------------
% RESULTADOS
% ------------------------------------------------------------------------
\section{Resultados}
Com a finalização dos experimentos, foi possível comprovar que algumas condições específicas são imprescindíveis para fazer uma informação vazar através da exploração da execução especulativa. A variação 01 da \emph{Spectre} possuí circunstâncias limitadoras bem definidas e específicas, que não estavam fundamentalmente claros na literatura.

A \emph{PoC} da \emph{Spectre} acontece em contextos iguais de forma absoluta e em contextos diferentes de forma relativa. Quando \citeonline{Kocher2018Spectre} cita que "um atacante pode ler memória arbitrária de um outro contexto", ele quer dizer que esse outro contexto é subcontexto de um pai. E tanto o atacante, quanto a vítima, pertencem ao mesmo subcontexto pai. Esse é um dos limites principais de exploração da \emph{Spectre}.

Além de estarem no mesmo contexto pai, vítima e atacante devem compartilhar de determinada região de memória. Isto é um limite também verdadeiro. Conforme os experimentos feitos, não existe possibilidade de manipulação da memória cache, nem de cálculo de endereços de variáveis, cujo atacante não conhece ou não tem acesso. Seria necessário que a própria vítima mostrasse os endereços físicos e, mesmo assim, o sistema operacional negaria as solicitações de \emph{flush} destes endereços físicos.

% ------------------------------------------------------------------------
% CONCLUSÃO
% ------------------------------------------------------------------------
\section{Conclusão}
As falhas de segurança nos computadores muitas vezes preocupam as pessoas. Principalmente quando se trata da segurança de informações pessoais. As vulnerabilidades de arquitetura descobertas por Jann Horn, colocaram muitas preocupações na comunidade de pesquisa, a respeito de quão seguros são os processadores que o mundo inteiro está usando.

Por mais preocupantes que estas falhas as vezes pareçam ser, ou até mesmo, por mais vulneráveis que os computadores possam parecer, nem sempre o que se pode ganhar de vantagem com essas falhas é algo relevante. No caso da primeira variação da \emph{Spectre} (que é uma falha de nível micro arquitetural, aparentemente sem correção fácil), as condições necessárias para exploração são tão específicas que, dificilmente, alguém conseguiria explorar remotamente esta vulnerabilidade para realizar um vazamento de informações sigilosas.

Estas condições foram elucidadas através de testes, que provaram que a primeira variação da \emph{Spectre} é uma vulnerabilidade que acontece em um trecho de código padronizado e específico (uma estrutura de verificação condicional de limites). E para ser explorada, esse mesmo código vulnerável precisa estar no mesmo contexto do atacante. Que por sua vez, precisa ter permissões de acesso aos endereços de todas variáveis da vítima deste código vulnerável.

Estas explorações podem acontecer em \emph{browsers}, máquinas virtuais, servidores de serviços na nuvem (\emph{cloud computing}) e outras aplicações que executam códigos em contextos compartilhados. Porém, neste artigo foi estudado a possibilidade de vazamento com código nativo, sendo que o código necessário para estas explorações citadas são feitas com outros tipos de código, utilizando o mesmo conceito. Com código nativo, esta exploração não acontece de forma fácil. Seria preciso ter um acesso prévio a máquina que se quer extrair as informações e mesmo assim, dificilmente se conseguiria uma informação útil. Por isso, não é possível explorar informações sensíveis em \emph{softwares} cujas estruturas sejam totalmente desconhecidas, através da indução da execução especulativa errônea, utilizando código nativo.

Nesta pesquisa, foi possível delimitar as circunstâncias necessárias para a exploração da primeira variação da \emph{Spectre}. Além de realizar comparações históricas fiéis as descobertas dos pesquisadores envolvidos. Esta pesquisa também possibilitou a sistematização e segmentação de um conhecimento relativamente novo e com um alto potencial de relevância acadêmica. O que pode auxiliar pesquisadores em revisões da literatura ou até mesmo em técnicas de pesquisa para produção de novos artigos sobre o assunto.

O resultado negativo obtido nos testes de extração de dados de outros contextos pode ser considerado como uma limitação desta pesquisa ou uma barreira técnica. Tal limitação poderá ser superada por outras pesquisas que apresentem resultados positivos no futuro. Da mesma maneira, as explicações para os testes negativos podem ser extraídas com mais detalhes em pesquisas futuras que considerem outras fontes de materiais não processados.

Como perspectiva de pesquisas futuras, pretende-se realizar novos testes para comprovar as limitações da primeira variação da \emph{Spectre}. Porém, com foco em outras formas de exploração de informações sensíveis. Não somente limitando-se a exploração de código nativo, como o feito nesta pesquisa. Mas, explorando outras possibilidades já elucidadas por Jann Horn, como por exemplo os interpretadores de código em tempo real (JIT, \emph{just-in-time}).

% ------------------------------------------------------------------------
% AGRADECIMENTOS
% ------------------------------------------------------------------------
\section*{Agradecimentos}
Agradecimentos a \textbf{Anderson} Eduardo Nascimento (Co-Fundador na \emph{Allele Security Intelligence}\footnote{Acesso em: \url{https://allelesecurity.com.br}}) pelas correções de conceitos, delimitação do escopo de pesquisa e informações pertinentes ao entendimento dos conceitos fundamentais, além de recomendações de matérias relacionadas. Também a \textbf{Alan} Messias Cordeiro (alacerda), pela ajuda na escolha do tema e matérias relacionadas. E \textbf{Maycon} Vitali (Fundador da \emph{Hack N' Roll}) pelas sugestões de material relacionado.

% ------------------------------------------------------------------------
% REFERÊNCIAS
% ------------------------------------------------------------------------
\bibliographystyle{IEEEtran}
\bibliography{references}
\end{document}